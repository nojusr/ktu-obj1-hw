\documentclass{article}

\usepackage{amsmath}
\usepackage[margin=2cm,a4paper]{geometry} 
\usepackage[T1]{fontenc}
\usepackage[english,lithuanian]{babel}
\usepackage[unicode,hidelinks]{hyperref}
\usepackage{graphicx}
\usepackage{enumitem}
\usepackage[final]{pdfpages}
\usepackage{titlesec}
\usepackage{textcomp}
\usepackage{lmodern, newunicodechar}
\usepackage{minted}
\usepackage{fontspec}

% change subsection formatting
\titleformat*{\subsection}{\large\bfseries}

% small function to set margins for specific elements
\def\changemargin#1#2{\list{}{\rightmargin#2\leftmargin#1}\item[]}
\let\endchangemargin=\endlist 

% set the default fonts
\defaultfontfeatures{Scale = MatchLowercase}
\setmainfont{Liberation Serif}[Scale = 1.0]
\setmonofont{Courier New}

\begin{document}

% redefine matrix command for easily changeable matrix vspacing
\makeatletter
\renewcommand*\env@matrix[1][\arraystretch]{%
  \edef\arraystretch{#1}%
  \hskip -\arraycolsep
  \let\@ifnextchar\new@ifnextchar
  \array{*\c@MaxMatrixCols c}}
\makeatother

% Redefinition of ToC command to get centered heading
\makeatletter
\renewcommand\tableofcontents{%
  \null\hfill\textbf{\Large\contentsname}\hfill\null\par
  \@mkboth{\MakeUppercase\contentsname}{\MakeUppercase\contentsname}%
  \@starttoc{toc}%
}
\makeatother

% minted settings
\setminted{
    breaklines=true,
    fontsize=\small
}




%-------------------------------------
%TITLE PAGE
%-------------------------------------

% ktu logo
\begin{figure}[!ht]
    \centering
    \includegraphics[width=2.5cm]{Assets/ktu-ikona.png}
\end{figure}

\vspace{-0.8cm}

\begin{center}
    \Large\textbf{Kauno technologijos universitetas}
\end{center}

\vspace{-0.6cm}

\begin{center}
    \large{Informatikos fakultetas}
\end{center}

\vspace{5cm}

\begin{center}
    \LARGE\textbf{Objektinis programavimas I (P175B118)}
\end{center}

\vspace{-0.5cm}

\begin{center}
    \Large{Laboratorinių darbų ataskaita}
\end{center}


\vspace{2.5cm}


\begin{changemargin}{7.5cm}{0cm}

    \begin{itemize}
        \item[]\large\textbf{Nojus Raškevičius, IFF-0/6}
        \vspace{0.2cm} 
        \item[]\large{Studentas}
    \end{itemize}

    \vspace{1cm}

    \begin{itemize}
        \item[]\large\textbf{Prof. Vacius Jusas}
        \vspace{0.2cm} 
        \item[]\large{Dėstytojas}
    \end{itemize}

\end{changemargin}

\vspace*{\fill}

\begin{center}
    \large\textbf{Kaunas 2020}
\end{center}

\pagebreak
\newpage

%-----------------------------------------------------------------------------------------------------------------
%CONTENTS
%-----------------------------------------------------------------------------------------------------------------

\tableofcontents

\newpage

%-----------------------------------------------------------------------------------------------------------------
%L1
%-----------------------------------------------------------------------------------------------------------------

\section{Duomenų klasė}
\subsection{Darbo užduotis}
\textbf{Kompiuterinis žaidimas.} Kuriate „fantasy“ tipo kompiuterinį žaidimą. Duomenų faile turite informacija apie žaidimo herojus: vardas, rasė, klasė, gyvybės taškai, mana, žalos taškai, gynybos taškai, jėga, vikrumas, intelektas, ypatinga galia.
\begin{itemize}
    \item[•]Raskite daugiausiai gyvybės taškų turintį herojų, ekrane atspausdinkite jo vardą, rasę, klasę ir gyvybės taškų kiekį. Jei yra keli, spausdinkite visus.
    \item[•]Raskite žaidėją, kurio gynybos ir žalos taškų skirtumas yra mažiausias. Atspausdinkite informaciją apie žaidėją į ekraną. Jei yra keli, spausdinkite visus. 
    \item[•]Sudarykite visų herojų klasių sąrašą, klasių pavadinimus įrašykite į failą „Klasės.csv“ .
\end{itemize}


\subsection{Programos tekstas}

\inputminted{csharp}{Assets/L1/L1-rawtext.txt}

\subsection{Pradiniai duomenys ir rezultatai}

\textbf{Pradiniai duomenys:}

\inputminted{csharp}{Assets/L1/herojai.csv}

\normalsize
\begin{changemargin}{0cm}{0cm}
    \textbf{Rezultatai:}
\end{changemargin}

\footnotesize
\begin{verbatim}
Visi herojai:
┌──────────┬──────────────┬───────────┬────┬────┬────┬─────┬──┬──┬──┬────────────────┐
│Vardas    │Rasė          │Klasė      │G.t.│M.t.│Ž.t.│Gy.t.│J.│V.│I.│Ypat. galia     │
├──────────┼──────────────┼───────────┼────┼────┼────┼─────┼──┼──┼──┼────────────────┤
│Aloyzas   │Lokys         │Kunigas    │97  │72  │38  │35   │7 │7 │1 │Gerai gamina... │
├──────────┼──────────────┼───────────┼────┼────┼────┼─────┼──┼──┼──┼────────────────┤
│Aloyzas   │Žmogus        │Dainininkas│42  │69  │82  │73   │4 │5 │8 │Labai laimin... │
├──────────┼──────────────┼───────────┼────┼────┼────┼─────┼──┼──┼──┼────────────────┤
│Antanas   │Varliažmogis  │Kunigas    │18  │27  │25  │51   │1 │9 │9 │Labai gražio... │
├──────────┼──────────────┼───────────┼────┼────┼────┼─────┼──┼──┼──┼────────────────┤
│Antanas   │Varliažmogis  │Kunigas    │66  │87  │99  │25   │4 │2 │4 │Ugnies valdy... │
├──────────┼──────────────┼───────────┼────┼────┼────┼─────┼──┼──┼──┼────────────────┤
│Petras    │Driežažmogis  │Burtininkas│21  │55  │20  │17   │5 │1 │1 │Labai laimin... │
├──────────┼──────────────┼───────────┼────┼────┼────┼─────┼──┼──┼──┼────────────────┤
│Vardėnis  │Driežažmogis  │Magas      │59  │33  │40  │55   │0 │0 │8 │Labai gražio... │
├──────────┼──────────────┼───────────┼────┼────┼────┼─────┼──┼──┼──┼────────────────┤
│Motiejus  │Elfas         │Lankininkas│35  │44  │45  │40   │3 │7 │5 │Labai laimin... │
├──────────┼──────────────┼───────────┼────┼────┼────┼─────┼──┼──┼──┼────────────────┤
│Motiejus  │Tamsusis elfas│Kunigas    │86  │70  │26  │69   │8 │9 │7 │Labai gražio... │
├──────────┼──────────────┼───────────┼────┼────┼────┼─────┼──┼──┼──┼────────────────┤
│Motiejus  │Elfas         │Karys      │57  │71  │95  │51   │4 │7 │2 │Labai laimin... │
├──────────┼──────────────┼───────────┼────┼────┼────┼─────┼──┼──┼──┼────────────────┤
│Antanas   │Driežažmogis  │Riteris    │33  │6   │91  │80   │3 │4 │7 │Ugnies valdy... │
├──────────┼──────────────┼───────────┼────┼────┼────┼─────┼──┼──┼──┼────────────────┤
│Vardėnis  │Lokys         │Gydytojas  │18  │34  │43  │12   │1 │3 │5 │Ugnies valdy... │
├──────────┼──────────────┼───────────┼────┼────┼────┼─────┼──┼──┼──┼────────────────┤
│Motiejus  │Tamsusis elfas│Lankininkas│75  │63  │18  │22   │1 │2 │7 │Labai laimin... │
├──────────┼──────────────┼───────────┼────┼────┼────┼─────┼──┼──┼──┼────────────────┤
│Juozas    │Šuo           │Karys      │12  │56  │86  │38   │6 │7 │4 │Gerai gamina... │
├──────────┼──────────────┼───────────┼────┼────┼────┼─────┼──┼──┼──┼────────────────┤
│Vardėnis  │Varliažmogis  │Burtininkas│37  │47  │14  │75   │1 │6 │6 │Gerai gamina... │
├──────────┼──────────────┼───────────┼────┼────┼────┼─────┼──┼──┼──┼────────────────┤
│Motiejus  │Žmogus        │Riteris    │28  │23  │61  │81   │3 │1 │9 │Labai gražio... │
├──────────┼──────────────┼───────────┼────┼────┼────┼─────┼──┼──┼──┼────────────────┤
│Antanas   │Elfas         │Riteris    │56  │4   │16  │91   │9 │7 │5 │Labai laimin... │
├──────────┼──────────────┼───────────┼────┼────┼────┼─────┼──┼──┼──┼────────────────┤
│Juozas    │Žmogus        │Karys      │97  │17  │74  │69   │5 │7 │5 │Ugnies valdy... │
├──────────┼──────────────┼───────────┼────┼────┼────┼─────┼──┼──┼──┼────────────────┤
│Motiejus  │Varliažmogis  │Magas      │39  │74  │21  │31   │7 │5 │0 │Labai gražio... │
├──────────┼──────────────┼───────────┼────┼────┼────┼─────┼──┼──┼──┼────────────────┤
│Petras    │Žmogus        │Riteris    │46  │64  │92  │83   │8 │6 │9 │Labai laimin... │
├──────────┼──────────────┼───────────┼────┼────┼────┼─────┼──┼──┼──┼────────────────┤
│Vardėnis  │Elfas         │Karys      │61  │77  │81  │26   │1 │1 │9 │Gerai gamina... │
├──────────┼──────────────┼───────────┼────┼────┼────┼─────┼──┼──┼──┼────────────────┤
│Petras    │Driežažmogis  │Gydytojas  │92  │35  │99  │37   │3 │4 │0 │Labai gražio... │
├──────────┼──────────────┼───────────┼────┼────┼────┼─────┼──┼──┼──┼────────────────┤
│Juozas    │Driežažmogis  │Riteris    │32  │60  │48  │83   │7 │6 │6 │Labai laimin... │
├──────────┼──────────────┼───────────┼────┼────┼────┼─────┼──┼──┼──┼────────────────┤
│Aloyzas   │Driežažmogis  │Lankininkas│65  │21  │98  │68   │6 │6 │9 │Labai gražio... │
├──────────┼──────────────┼───────────┼────┼────┼────┼─────┼──┼──┼──┼────────────────┤
│Juozas    │Driežažmogis  │Gydytojas  │91  │51  │63  │23   │4 │5 │4 │Ugnies valdy... │
├──────────┼──────────────┼───────────┼────┼────┼────┼─────┼──┼──┼──┼────────────────┤
│Juozas    │Zombis        │Gydytojas  │93  │95  │82  │23   │9 │9 │5 │Labai gražio... │
└──────────┴──────────────┴───────────┴────┴────┴────┴─────┴──┴──┴──┴────────────────┘
Herojai su didžiausiu kiekiu gyvybės taškų:
┌──────────────────┬──────────────────┬──────────────────┬──────────────────┐
│ Vardas           │ Rasė             │ Klasė            │ Gyvybės t.       │
├──────────────────┼──────────────────┼──────────────────┼──────────────────┤
│ Aloyzas          │ Lokys            │ Kunigas          │ 97               │
├──────────────────┼──────────────────┼──────────────────┼──────────────────┤
│ Juozas           │ Žmogus           │ Karys            │ 97               │
└──────────────────┴──────────────────┴──────────────────┴──────────────────┘
Herojai su mažiausiu skirtumu tarp žalos ir gynybos taškų:
┌──────────┬──────────────┬───────────┬────┬────┬────┬─────┬──┬──┬──┬────────────────┐
│Vardas    │Rasė          │Klasė      │G.t.│M.t.│Ž.t.│Gy.t.│J.│V.│I.│Ypat. galia     │
├──────────┼──────────────┼───────────┼────┼────┼────┼─────┼──┼──┼──┼────────────────┤
│Aloyzas   │Lokys         │Kunigas    │97  │72  │38  │35   │7 │7 │1 │Gerai gamina... │
├──────────┼──────────────┼───────────┼────┼────┼────┼─────┼──┼──┼──┼────────────────┤
│Petras    │Driežažmogis  │Burtininkas│21  │55  │20  │17   │5 │1 │1 │Labai laimin... │
└──────────┴──────────────┴───────────┴────┴────┴────┴─────┴──┴──┴──┴────────────────┘
\end{verbatim}

\normalsize
\begin{changemargin}{0cm}{0cm}
    \textbf{Klasės.csv:}
\end{changemargin}
\inputminted{csharp}{Assets/L1/klases.csv}

\subsection{Dėstytojo pastabos}
\begin{itemize}
    \item[•]Ataskaitos pavadinimas buvo ne pagal taisykles
    \item[•]Dėstytotojo pareigos buvo ne tos
    \item[•]Metodai nebuvo tinkamai bei visiškai aprašyti
    \item[•]Nėra pateikti komentarai apie pradinių duomenų prasmę
\end{itemize}

\changemargin{0cm}{0cm}\textbf{Testo rezultatai: 2/3}\newline
\textbf{Savarankiško darbo rezultatai: 0/1}\newline
\textbf{Laboratorinio darbo įvertinimas: 6/7}\newline
\textbf{Galutinis įvertinimas: 8}\newline

\newpage

%-----------------------------------------------------------------------------------------------------------------
%L2
%-----------------------------------------------------------------------------------------------------------------


\section{Skaičiavimų klasė}

\subsection{Darbo užduotis}
\textbf{Krepšinio rinktinė.} urite  ne  tik  šių,  bet  ir  vienų  ankstesniųjų  metų  į  stovyklas  pakviestų krepšininkų sąrašus. Keičiasi duomenų failų formatas. Pirmoje eilutėje metai, antroje –stovyklos pradžios data, trečioje –stovyklos pabaigos data. Toliau informacija apie krepšininkus pateikta tokiu pačiu formatu kaip L1 užduotyje.
\begin{itemize}
    \item[•]Sudarykite visų puolėjų, dalyvavusių rinktinės stovyklose, sąrašą ir ekrane atspausdinkite jų vardus, pavardes bei ūgį.
    \item[•]Raskite aukščiausiąkrepšininką, ir ekrane atspausdinkite jo vardą, pavardę bei amžių. Jei yra keli, spausdinkite visus.
    \item[•]Sudarykite sąrašą klubų, kuriuose žaidė kandidatai į rinktinę, ir įrašykite į failą „Klubai.csv“.
\end{itemize}

\subsection{Programos tekstas}
\inputminted{csharp}{Assets/L2/L2-rawtext.txt}
\newpage

\subsection{Pradiniai duomenys ir rezultatai}

\subsubsection{Pirmas tikrinimas}
\begin{changemargin}{0cm}{0cm}
    \textbf{krepsininkai-2020.csv:}
\end{changemargin}

\inputminted{csharp}{Assets/L2/test1-input1.txt}

\begin{changemargin}{0cm}{0cm}
    \textbf{krepsininkai-2019.csv:}
\end{changemargin}

\inputminted{csharp}{Assets/L2/test1-input2.txt}


\begin{changemargin}{0cm}{0cm}
    Šitie duomenys yra skirti bendram tikrinimui atlikti. Išvestyje turėtų būti tik vienas aukščiausias žaidėjas,
    (Juozas Pavardėnis), turėtų būti 7 puolėjai, turėtų būti 4 unikalūs klubai (Šaulys, L. Rytas, Žalgiris, Kruojos).
\end{changemargin}

\begin{changemargin}{0cm}{0cm}
    \textbf{Programos išvestis:}
\end{changemargin}


\inputminted[fontsize=\footnotesize]{csharp}{Assets/L2/test1-output.txt}

\footnotesize
\begin{verbatim}
Visi puolėjai:
┌──────────┬──────────────┬───┐
│Vardas    │Pavardė       │Au.│
├──────────┼──────────────┼───┤
│Aloyzas   │Valančiūnas   │182│
├──────────┼──────────────┼───┤
│Petras    │Pavardėnis    │197│
├──────────┼──────────────┼───┤
│Antanas   │Žukauskas     │198│
├──────────┼──────────────┼───┤
│Antanas   │Jasikevičius  │195│
├──────────┼──────────────┼───┤
│Aloyzas   │Valančiūnas   │195│
├──────────┼──────────────┼───┤
│Antanas   │Valančiūnas   │187│
├──────────┼──────────────┼───┤
│Juozas    │Pavardėnis    │207│
└──────────┴──────────────┴───┘
Aukščiausi žaidėjai:
┌──────────┬──────────────┬───┐
│Vardas    │Pavardė       │Au.│
├──────────┼──────────────┼───┤
│Juozas    │Pavardėnis    │207│
└──────────┴──────────────┴───┘
\end{verbatim}
\normalsize

\begin{changemargin}{0cm}{0cm}
    \textbf{Klubai.csv:}
\end{changemargin}

\footnotesize
\begin{verbatim}
L. Rytas
Šaulys
Žalgiris
Kruojos
\end{verbatim}
\normalsize

\subsubsection{Antras tikrinimas}
\begin{changemargin}{0cm}{0cm}
    \textbf{krepsininkai-2020.csv:}
\end{changemargin}

\inputminted{csharp}{Assets/L2/test2-input1.txt}

\begin{changemargin}{0cm}{0cm}
    \textbf{krepsininkai-2019.csv:}
\end{changemargin}

\inputminted{csharp}{Assets/L2/test2-input2.txt}


\begin{changemargin}{0cm}{0cm}
    Šie įvesties duomenys yra skirti tikrinti „Klubai.csv“ išvestį.
    Yra įvesti tokie patys klubai kaip ir pirmajame tikrinime, tačiau jie yra pakeisti taip,
    kad žaidėjai yra priimami tik iš „Žalgirio“ ir „Šaulio“ klubų.
    Kitaip tariant, Į „Klubai.csv“ turėtų buti išvesti tik du klubai: „Šaulys“ ir „Žalgiris“.
\end{changemargin}

\begin{changemargin}{0cm}{0cm}
    \textbf{Programos išvestis:}
\end{changemargin}


\inputminted[fontsize=\footnotesize]{csharp}{Assets/L2/test2-output.txt}

\footnotesize
\begin{verbatim}
Visi puolėjai:
┌──────────┬──────────────┬───┐
│Vardas    │Pavardė       │Au.│
├──────────┼──────────────┼───┤
│Vardėnis  │Sabonis       │193│
├──────────┼──────────────┼───┤
│Antanas   │Valančiūnas   │191│
├──────────┼──────────────┼───┤
│Juozas    │Pavardėnis    │194│
├──────────┼──────────────┼───┤
│Petras    │Sabonis       │191│
└──────────┴──────────────┴───┘
Aukščiausi žaidėjai:
┌──────────┬──────────────┬───┐
│Vardas    │Pavardė       │Au.│
├──────────┼──────────────┼───┤
│Petras    │Pavardėnis    │209│
├──────────┼──────────────┼───┤
│Antanas   │Sabonis       │209│
└──────────┴──────────────┴───┘
\end{verbatim}
\normalsize

\begin{changemargin}{0cm}{0cm}
    \textbf{Klubai.csv:}
\end{changemargin}

\footnotesize
\begin{verbatim}
Šaulys
Žalgiris
\end{verbatim}
\normalsize

\subsection{Dėstytojo pastabos}
\begin{itemize}
    \item[•]Dėstytojo pastabose būtina įrašyti ir pažymius.
    \item[•]O taip negalima publicList GetAllPlayers(). Tik vieną grąžinti!
    \item[•]Vienu ciklu - visi aukščiausi. Įdomu!
    \item[•]Viską į vieną objektą - reg.AddRange()! Negerai! Bet, šiaip, geras darbas. 
\end{itemize}

\changemargin{0cm}{0cm}\textbf{Testo rezultatai: 1/3}\newline
\textbf{Savarankiško darbo rezultatai: 0/1}\newline
\textbf{Laboratorinio darbo įvertinimas: 7/7}\newline
\textbf{Galutinis įvertinimas: 8}\newline

\newpage

%-----------------------------------------------------------------------------------------------------------------
%L3
%-----------------------------------------------------------------------------------------------------------------

\section{Konteineris}
\subsection{Darbo užduotis}
\textbf{Automobilių parkas.} Įmonė UAB „Žaibas“ turi du filialus. Keičiasi duomenų formatas. Pirmoje eilutėje miestas, antroje –adresas, trečioje –telefonas.  Toliau  informacija  apie  automobilius  pateikta  tokiu pačiu formatu kaip L1 užduotyje.

\begin{itemize}
    \item[•]Raskite, kuriame filiale automobiliai seniausi (vidutinis automobilio amžius didžiausias). Filialo duomenis atspausdinkite ekrane.
    \item[•]Raskite naujausią automobilį. Atspausdinkite  ekranevisus  jo  duomenis.Jei  yra  keli,  spausdinkite visus.
    \item[•]Pastebėjote, kad duomenų failuose įsivėlė klaidų ir kai kurie automobiliai yra priskirti abiem filialams vienu metu. Sudarykite tokiųautomobilių sąrašą ir į failą „Klaidos.csv“ įrašykite tų automobilių valstybinį numerį, modelį bei filialo, prie kurio jis priskirtas, pavadinimus.
    \item[•]Sudarykite automobilių, kuriems jau pasibaigęs techninės apžiūros galiojimas, arba liko mažiau nei mėnuo, sąrašą. Į failą „Apžiūra.csv“ įrašykite automobilio gamintoją, modelį, valstybinį numerį, techninės apžiūros galiojimo datą. Jei techninė apžiūra nebegalioja, atitinkamoje eilutėje įrašykite žodį „SKUBIAI“.Surikiuokite automobilius pagal gamintojus, modelius ir valstybinį numerį
\end{itemize}

\subsection{Programos tekstas}

\inputminted{csharp}{Assets/L3/L3-rawtext.txt}

\subsection{Pradiniai duomenys ir rezultatai}

\subsubsection{Pirmas tikrinimas}

Šie duomenys yra skirti tikrinti bendrą programos veiklą ir įsitikinti, kad apžiūros (Apžiūra1.csv, Apžiūra2.csv)
failai tinkamai sudėlioja visus automobilius pagal nurodytą eilės tvarką.
Pirmas registras (filialas) turi senesnius automobilius.

\begin{changemargin}{0cm}{0cm}
    \textbf{Kaunas.csv:}
\end{changemargin}

\inputminted{csharp}{Assets/L3/test1-input1.txt}

\begin{changemargin}{0cm}{0cm}
    \textbf{Siauliai.csv:}
\end{changemargin}

\inputminted{csharp}{Assets/L3/test1-input2.txt}

\textbf{Programos išvestis}

\footnotesize
\begin{verbatim}
Pirmame Filiale (registre) yra senesni automobiliai.
UAB „Žaibas Kaunas“ priklausantys automobiliai:
┌───────┬────────────┬────────┬─────┬────┬───────────────┬──────────┬────────────┐
│Val. ID│Gamintojas  │Modelis │Metai│Mėn.│T.A. gal. data │Kuras     │Vid. sąnaud.│
├───────┼────────────┼────────┼─────┼────┼───────────────┼──────────┼────────────┤
│JLS465 │BMW         │X7      │2004 │3   │1/23/2021      │dyzelis   │4           │
│AKF435 │Audi        │A4      │2005 │4   │2/23/2021      │benzinas  │5.3         │
│DSF342 │BMW         │3       │1990 │11  │2/24/2021      │elektrine │3           │
│DDD999 │BMW         │V70     │2009 │9   │9/29/2021      │dyzelis   │4.8         │
│UIF805 │Audi        │A80     │2000 │12  │2/27/2021      │benzinas  │9.3         │
│JYY875 │Audi        │A100    │1990 │2   │1/27/2019      │dyzelis   │6.3         │
│DKI234 │Passat      │B6      │1996 │5   │3/25/2019      │benzinas  │5.6         │
│FCX456 │Ford        │Focus   │2016 │4   │5/21/2019      │dyzelis   │7.9         │
│DFD499 │BMW         │S60     │2016 │4   │9/19/2021      │dyzelis   │8           │
│JXF875 │Audi        │A80     │1990 │2   │1/27/2021      │benzinas  │6.3         │
│DID234 │BMW         │B6      │1996 │5   │3/25/2021      │benzinas  │5.6         │
│FVW456 │Ford        │Focus   │2016 │4   │5/21/2021      │benzinas  │7.9         │
│DFD759 │Volvo       │S60     │2006 │1   │9/19/2021      │dyzelis   │10.5        │
│KHT367 │Volvo       │S60     │2013 │7   │10/19/2021     │benzinas  │8           │
│FGF875 │Audi        │A80     │1994 │5   │5/27/2021      │dujos     │7.3         │
│OLF279 │Audi        │A80     │1998 │6   │5/27/2021      │dujos     │7.3         │
└───────┴────────────┴────────┴─────┴────┴───────────────┴──────────┴────────────┘

Pirmas registras:
UAB „Žaibas Kaunas“ priklausantys automobiliai:
┌───────┬────────────┬────────┬─────┬────┬───────────────┬──────────┬────────────┐
│Val. ID│Gamintojas  │Modelis │Metai│Mėn.│T.A. gal. data │Kuras     │Vid. sąnaud.│
├───────┼────────────┼────────┼─────┼────┼───────────────┼──────────┼────────────┤
│JLS465 │BMW         │X7      │2004 │3   │1/23/2021      │dyzelis   │4           │
│AKF435 │Audi        │A4      │2005 │4   │2/23/2021      │benzinas  │5.3         │
│DSF342 │BMW         │3       │1990 │11  │2/24/2021      │elektrine │3           │
│DDD999 │BMW         │V70     │2009 │9   │9/29/2021      │dyzelis   │4.8         │
│UIF805 │Audi        │A80     │2000 │12  │2/27/2021      │benzinas  │9.3         │
│JYY875 │Audi        │A100    │1990 │2   │1/27/2019      │dyzelis   │6.3         │
│DKI234 │Passat      │B6      │1996 │5   │3/25/2019      │benzinas  │5.6         │
│FCX456 │Ford        │Focus   │2016 │4   │5/21/2019      │dyzelis   │7.9         │
│DFD499 │BMW         │S60     │2016 │4   │9/19/2021      │dyzelis   │8           │
│JXF875 │Audi        │A80     │1990 │2   │1/27/2021      │benzinas  │6.3         │
│DID234 │BMW         │B6      │1996 │5   │3/25/2021      │benzinas  │5.6         │
│FVW456 │Ford        │Focus   │2016 │4   │5/21/2021      │benzinas  │7.9         │
│DFD759 │Volvo       │S60     │2006 │1   │9/19/2021      │dyzelis   │10.5        │
│KHT367 │Volvo       │S60     │2013 │7   │10/19/2021     │benzinas  │8           │
│FGF875 │Audi        │A80     │1994 │5   │5/27/2021      │dujos     │7.3         │
│OLF279 │Audi        │A80     │1998 │6   │5/27/2021      │dujos     │7.3         │
└───────┴────────────┴────────┴─────┴────┴───────────────┴──────────┴────────────┘
Naujausias(-i) automobilis(-iai):
┌───────┬────────────┬────────┬─────┬────┬───────────────┬──────────┬────────────┐
│Val. ID│Gamintojas  │Modelis │Metai│Mėn.│T.A. gal. data │Kuras     │Vid. sąnaud.│
├───────┼────────────┼────────┼─────┼────┼───────────────┼──────────┼────────────┤
│FCX456 │Ford        │Focus   │2016 │4   │5/21/2019      │dyzelis   │7.9         │
│DFD499 │BMW         │S60     │2016 │4   │9/19/2021      │dyzelis   │8           │
│FVW456 │Ford        │Focus   │2016 │4   │5/21/2021      │benzinas  │7.9         │
└───────┴────────────┴────────┴─────┴────┴───────────────┴──────────┴────────────┘

Antras registras:
UAB „Žaibas Šiauliai“ priklausantys automobiliai:
┌───────┬────────────┬────────┬─────┬────┬───────────────┬──────────┬────────────┐
│Val. ID│Gamintojas  │Modelis │Metai│Mėn.│T.A. gal. data │Kuras     │Vid. sąnaud.│
├───────┼────────────┼────────┼─────┼────┼───────────────┼──────────┼────────────┤
│LYK465 │BMW         │X7      │2017 │3   │1/23/2019      │dyzelis   │4           │
│ASD435 │Audi        │A6      │2005 │4   │2/23/2019      │benzinas  │5.3         │
│LLL342 │Mazda       │3       │1990 │11  │2/24/2018      │elektrine │3           │
│DYY999 │Volvo       │V70     │2009 │9   │9/29/2018      │dyzelis   │4.8         │
│UPP805 │Audi        │A6      │2000 │12  │2/27/2019      │benzinas  │9.3         │
│DOO499 │Volvo       │S80     │2012 │4   │9/19/2018      │dyzelis   │8           │
│JYY875 │Audi        │A100    │1990 │2   │1/27/2019      │dyzelis   │6.3         │
│DKI234 │Passat      │B6      │1996 │5   │3/25/2019      │benzinas  │5.6         │
│FCX456 │Ford        │Focus   │2016 │4   │5/21/2019      │dyzelis   │7.9         │
│DKL759 │Volvo       │S80     │2008 │1   │9/19/2018      │dyzelis   │10.5        │
│OOO367 │Volvo       │S60     │2012 │7   │9/19/2019      │benzinas  │8           │
│PPP875 │Audi        │A6      │2018 │5   │5/27/2019      │dyzelis   │7.3         │
│LZM279 │Audi        │A7      │2006 │6   │5/27/2019      │dujos     │7.3         │
│JXF875 │Audi        │A80     │1990 │2   │1/27/2021      │benzinas  │6.3         │
└───────┴────────────┴────────┴─────┴────┴───────────────┴──────────┴────────────┘
Naujausias(-i) automobilis(-iai):
┌───────┬────────────┬────────┬─────┬────┬───────────────┬──────────┬────────────┐
│Val. ID│Gamintojas  │Modelis │Metai│Mėn.│T.A. gal. data │Kuras     │Vid. sąnaud.│
├───────┼────────────┼────────┼─────┼────┼───────────────┼──────────┼────────────┤
│PPP875 │Audi        │A6      │2018 │5   │5/27/2019      │dyzelis   │7.3         │
└───────┴────────────┴────────┴─────┴────┴───────────────┴──────────┴────────────┘
\end{verbatim}
\normalsize

\textbf{Klaidos.csv:}
\inputminted{csharp}{Assets/L3/klaidos1.csv}

\textbf{Apžiūra1.csv:}
\inputminted{csharp}{Assets/L3/apziura1-1.csv}

\textbf{Apžiūra2.csv:}
\inputminted{csharp}{Assets/L3/apziura1-2.csv}

\subsubsection{Antras tikrinimas:}

Šie duomenys yra skirti tikrinti ar programa išveda kelias naujausias mašinas ir ar programa tinkamai išveda į Klaidos.csv.
Į klaidas turėtų buti išvesti automobiliai PPP875, ASD435 ir UIF805.
Naujausi automobiliai pirmame regsitre turėtų būti ABC123, ABC124, ABC125.
Naujausias automobilis antrame registre turėtų būti PPP875.


\begin{changemargin}{0cm}{0cm}
    \textbf{Kaunas.csv:}
\end{changemargin}

\inputminted{csharp}{Assets/L3/test2-input1.txt}

\begin{changemargin}{0cm}{0cm}
    \textbf{Siauliai.csv:}
\end{changemargin}

\inputminted{csharp}{Assets/L3/test2-input2.txt}



\textbf{Programos išvestis:}

\footnotesize
\begin{verbatim}
Antrame Filiale (registre) yra senesni automobiliai.
UAB „Žaibas Šiauliai“ priklausantys automobiliai:
┌───────┬────────────┬────────┬─────┬────┬───────────────┬──────────┬────────────┐
│Val. ID│Gamintojas  │Modelis │Metai│Mėn.│T.A. gal. data │Kuras     │Vid. sąnaud.│
├───────┼────────────┼────────┼─────┼────┼───────────────┼──────────┼────────────┤
│LYK465 │BMW         │X7      │2017 │3   │1/23/2019      │dyzelis   │4           │
│ASD435 │Audi        │A6      │2016 │1   │2/23/2019      │benzinas  │5.3         │
│LLL342 │Mazda       │3       │1990 │11  │2/24/2018      │elektrine │3           │
│DYY999 │Volvo       │V70     │2009 │9   │9/29/2018      │dyzelis   │4.8         │
│UPP805 │Audi        │A6      │2000 │12  │2/27/2019      │benzinas  │9.3         │
│DOO499 │Volvo       │S80     │2012 │4   │9/19/2018      │dyzelis   │8           │
│JYY875 │Audi        │A100    │1990 │2   │1/27/2019      │dyzelis   │6.3         │
│DKI234 │Passat      │B6      │1996 │5   │3/25/2019      │benzinas  │5.6         │
│FCX456 │Ford        │Focus   │2016 │4   │5/21/2019      │dyzelis   │7.9         │
│DKL759 │Volvo       │S80     │2008 │1   │9/19/2018      │dyzelis   │10.5        │
│OOO367 │Volvo       │S60     │2012 │7   │9/19/2019      │benzinas  │8           │
│PPP875 │Audi        │A6      │2018 │5   │5/27/2019      │dyzelis   │7.3         │
│LZM279 │Audi        │A7      │2006 │6   │5/27/2019      │dujos     │7.3         │
│UIF805 │Audi        │A80     │2000 │12  │2/27/2021      │benzinas  │9.3         │
│JXF875 │Audi        │A80     │1990 │2   │1/27/2021      │benzinas  │6.3         │
└───────┴────────────┴────────┴─────┴────┴───────────────┴──────────┴────────────┘

Pirmas registras:
UAB „Žaibas Kaunas“ priklausantys automobiliai:
┌───────┬────────────┬────────┬─────┬────┬───────────────┬──────────┬────────────┐
│Val. ID│Gamintojas  │Modelis │Metai│Mėn.│T.A. gal. data │Kuras     │Vid. sąnaud.│
├───────┼────────────┼────────┼─────┼────┼───────────────┼──────────┼────────────┤
│ABC123 │BMW         │X7      │2020 │5   │2/27/2021      │dyzelis   │4           │
│AKF435 │Audi        │A4      │2005 │4   │2/23/2021      │benzinas  │5.3         │
│ASD435 │Audi        │A6      │2020 │1   │2/23/2019      │benzinas  │5.3         │
│DSF342 │BMW         │3       │1990 │11  │2/24/2021      │elektrine │3           │
│DDD999 │BMW         │V70     │2009 │9   │9/29/2021      │dyzelis   │4.8         │
│UIF805 │Audi        │A80     │2000 │12  │2/27/2021      │benzinas  │9.3         │
│DFD499 │BMW         │S60     │2016 │4   │9/19/2021      │dyzelis   │8           │
│DID234 │BMW         │B6      │1996 │5   │3/25/2021      │benzinas  │5.6         │
│FVW456 │Ford        │Focus   │2016 │4   │5/21/2021      │benzinas  │7.9         │
│ABC125 │Volvo       │S60     │2020 │5   │9/19/2021      │dyzelis   │10.5        │
│KHT367 │Volvo       │S60     │2013 │7   │10/19/2021     │benzinas  │8           │
│ABC124 │Audi        │A80     │2020 │5   │5/27/2021      │dujos     │7.3         │
│OLF279 │Audi        │A80     │1998 │6   │5/27/2021      │dujos     │7.3         │
│PPP875 │Audi        │A6      │2018 │5   │5/27/2019      │dyzelis   │7.3         │
└───────┴────────────┴────────┴─────┴────┴───────────────┴──────────┴────────────┘
Naujausias(-i) automobilis(-iai):
┌───────┬────────────┬────────┬─────┬────┬───────────────┬──────────┬────────────┐
│Val. ID│Gamintojas  │Modelis │Metai│Mėn.│T.A. gal. data │Kuras     │Vid. sąnaud.│
├───────┼────────────┼────────┼─────┼────┼───────────────┼──────────┼────────────┤
│ABC123 │BMW         │X7      │2020 │5   │2/27/2021      │dyzelis   │4           │
│ABC125 │Volvo       │S60     │2020 │5   │9/19/2021      │dyzelis   │10.5        │
│ABC124 │Audi        │A80     │2020 │5   │5/27/2021      │dujos     │7.3         │
└───────┴────────────┴────────┴─────┴────┴───────────────┴──────────┴────────────┘

Antras registras:
UAB „Žaibas Šiauliai“ priklausantys automobiliai:
┌───────┬────────────┬────────┬─────┬────┬───────────────┬──────────┬────────────┐
│Val. ID│Gamintojas  │Modelis │Metai│Mėn.│T.A. gal. data │Kuras     │Vid. sąnaud.│
├───────┼────────────┼────────┼─────┼────┼───────────────┼──────────┼────────────┤
│LYK465 │BMW         │X7      │2017 │3   │1/23/2019      │dyzelis   │4           │
│ASD435 │Audi        │A6      │2016 │1   │2/23/2019      │benzinas  │5.3         │
│LLL342 │Mazda       │3       │1990 │11  │2/24/2018      │elektrine │3           │
│DYY999 │Volvo       │V70     │2009 │9   │9/29/2018      │dyzelis   │4.8         │
│UPP805 │Audi        │A6      │2000 │12  │2/27/2019      │benzinas  │9.3         │
│DOO499 │Volvo       │S80     │2012 │4   │9/19/2018      │dyzelis   │8           │
│JYY875 │Audi        │A100    │1990 │2   │1/27/2019      │dyzelis   │6.3         │
│DKI234 │Passat      │B6      │1996 │5   │3/25/2019      │benzinas  │5.6         │
│FCX456 │Ford        │Focus   │2016 │4   │5/21/2019      │dyzelis   │7.9         │
│DKL759 │Volvo       │S80     │2008 │1   │9/19/2018      │dyzelis   │10.5        │
│OOO367 │Volvo       │S60     │2012 │7   │9/19/2019      │benzinas  │8           │
│PPP875 │Audi        │A6      │2018 │5   │5/27/2019      │dyzelis   │7.3         │
│LZM279 │Audi        │A7      │2006 │6   │5/27/2019      │dujos     │7.3         │
│UIF805 │Audi        │A80     │2000 │12  │2/27/2021      │benzinas  │9.3         │
│JXF875 │Audi        │A80     │1990 │2   │1/27/2021      │benzinas  │6.3         │
└───────┴────────────┴────────┴─────┴────┴───────────────┴──────────┴────────────┘
Naujausias(-i) automobilis(-iai):
┌───────┬────────────┬────────┬─────┬────┬───────────────┬──────────┬────────────┐
│Val. ID│Gamintojas  │Modelis │Metai│Mėn.│T.A. gal. data │Kuras     │Vid. sąnaud.│
├───────┼────────────┼────────┼─────┼────┼───────────────┼──────────┼────────────┤
│PPP875 │Audi        │A6      │2018 │5   │5/27/2019      │dyzelis   │7.3         │
└───────┴────────────┴────────┴─────┴────┴───────────────┴──────────┴────────────┘
\end{verbatim}
\normalsize

\pagebreak

\textbf{Klaidos.csv:}
\inputminted{csharp}{Assets/L3/klaidos2.csv}

\textbf{Apžiūra1.csv:}
\inputminted{csharp}{Assets/L3/apziura2-1.csv}

\textbf{Apžiūra2.csv:}
\inputminted{csharp}{Assets/L3/apziura2-2.csv}

\subsection{Dėstytojo pastabos}
\begin{itemize}
    \item[•]Buvo naudojamas ne tas rikiavimo būdas.
    \item[•]Be reikalo naudojami delegatai.
\end{itemize}

\changemargin{0cm}{0cm}\textbf{Testo rezultatai: 1/3}\newline
\textbf{Savarankiško darbo rezultatai: 0/1}\newline
\textbf{Laboratorinio darbo įvertinimas: 6/7}\newline
\textbf{Galutinis įvertinimas: 7}\newline

\newpage

%-----------------------------------------------------------------------------------------------------------------
%L4
%-----------------------------------------------------------------------------------------------------------------

\section{Teksto analizė ir redagavimas}
\subsection{Darbo užduotis}
Dviejuose  tekstiniuose  failuose \texttt{Knyga1.txt} ir \texttt{Knyga2.txt} duotas tekstas sudarytas iš žodžių,
atskirtų skyrikliais. Skyriklių aibė žinoma ir abejuose failuose yra ta pati. Analizuojant tekstus, didžiosios
ir mažosios raidės nesvarbios.Raskite ir spausdinkite faile \texttt{Rodikliai.txt}:
\begin{itemize}
    \item[•]ilgiausių žodžių, surikiuotų ilgio mažėjimo tvarka, kurie yra abejuose failuose, sąrašą (ne daugiau nei 10 žodžių) ir jų pasikartojimo skaičių kiekviename iš failų;
    \item[•]ilgiausių žodžių, surikiuotų ilgio mažėjimo tvarka, kurie yra tik faile \texttt{Knyga1.txt}, bet nėra faile \texttt{Knyga2.txt}, sąrašą (ne daugiau nei 10 žodžių) ir jų pasikartojimo skaičių;
\end{itemize}
Spausdinkite faile \texttt{ManoKnyga.txt} apjungtą tekstą, sudarytą pagal tokias taisykles:
\begin{itemize}
    \item[•]kopijuojamas  pirmojo  failo  tekstas  tol,  kol  sutinkamas pirmasis nenukopijuotas antrojo failo žodis arba pasiekiama failo pabaiga;
    \item[•]kopijuojamas antrojo failo tekstas tol, kol sutinkamas pirmasis nenukopijuotas pirmojo failo žodis arba pasiekiama failo pabaiga;
    \item[•]kartojama tol, kol pasiekiama abiejų failų pabaiga.
\end{itemize}

\subsection{Programos tekstas}

\inputminted{csharp}{Assets/L4/L4-rawtext.txt}

\subsection{Pradiniai duomenys ir rezultatai}

\subsubsection{Pirmas tikrinimas}

Šie pradiniai duomenys yra skirti bendrai programos veiklai tikrinti. \texttt{Knyga1.txt} turi įterpto lotyniško teksto,
kuris nėra įterptas į \texttt{Knyga2.txt}, todėl programos antroje lentelėje turėtų būti išvedami tik lotyniški žodžiai.

\begin{changemargin}{0cm}{0cm}
    \textbf{Knyga1.txt:}
\end{changemargin}

\inputminted{text}{Assets/L4/t1-k1.txt}

\vspace{0.5cm}

\begin{changemargin}{0cm}{0cm}
    \textbf{Knyga2.txt:}
\end{changemargin}


\inputminted{text}{Assets/L4/t1-k2.txt}

\begin{changemargin}{0cm}{0cm}
    \textbf{Programos išvestis:}
\end{changemargin}

\begin{verbatim}
┌──────┬──────┬────────────────────┐
│Knyga1│Knyga2│Žodis               │
├──────┼──────┼────────────────────┤
│1     │0     │reprehenderit       │
│2     │1     │tekstiniuose        │
│2     │1     │analizuojant        │
│1     │0     │exercitation        │
│2     │1     │skyrikliais         │
│1     │0     │consectetur         │
│1     │2     │sulygiuoti          │
│1     │2     │kiekvienos          │
│1     │2     │kiekvienas          │
│1     │2     │fiksuotoje          │
└──────┴──────┴────────────────────┘
┌──────┬────────────────────┐
│Knyga1│Žodis               │
├──────┼────────────────────┤
│1     │reprehenderit       │
│1     │exercitation        │
│1     │consectetur         │
│1     │adipiscing          │
│1     │incididunt          │
│1     │consequat           │
│1     │voluptate           │
│1     │excepteur           │
│1     │cupidatat           │
│1     │pariatur            │
└──────┴────────────────────┘
    
\end{verbatim}

\begin{changemargin}{0cm}{0cm}
    \textbf{ManoKnyga.txt:}
\end{changemargin}

\inputminted{text}{Assets/L4/t1-mk.txt}

\subsubsection{Antras tikrinimas}

Šie pradiniai duomenys yra skirti tikrinti \texttt{ManoKnyga.txt} išvestį. \texttt{Knyga1.txt} ir \texttt{Knyga2.txt} turi `Lorem ipsum` tekstą, su įterptais specifiniais
žodžiais, kurie kartojasi per abu minėtus tekstinius failus. \texttt{ManoKnyga.txt} turėtų turėti tokia informacijos eigą:
\begin{itemize}
    \item [•] \texttt{Knyga1.txt} bus kopijuojama iki pirmos eilutės pabaigos
    \item [•] toliau, bus kopijuojamos pirmos trys \texttt{Knyga2.txt} failo eilutės
    \item [•] toliau, bus kopijuojama nuo antros iki trečios \texttt{Knyga1.txt} failo eilučių
    \item [•] toliau, bus kopijuojama nuo ketvirtos \texttt{Knyga2.txt} eilutės iki \texttt{Knyga2.txt} pabaigos.
    \item [•] toliau, bus kopijuojama nuo ketvirtos \texttt{Knyga1.txt} eilutės iki \texttt{Knyga1.txt} pabaigos.
\end{itemize}

\begin{changemargin}{0cm}{0cm}
    \textbf{Knyga1.txt:}
\end{changemargin}

\inputminted{text}{Assets/L4/t2-k1.txt}

\vspace{0.5cm}

\begin{changemargin}{0cm}{0cm}
    \textbf{Knyga2.txt:}
\end{changemargin}


\inputminted{text}{Assets/L4/t2-k2.txt}

\begin{changemargin}{0cm}{0cm}
    \textbf{Programos išvestis:}
\end{changemargin}

\begin{verbatim}
┌──────┬──────┬────────────────────┐
│Knyga1│Knyga2│Žodis               │
├──────┼──────┼────────────────────┤
│1     │1     │reprehenderit       │
│1     │1     │exercitation        │
│1     │1     │consectetur         │
│1     │1     │adipiscing          │
│1     │1     │incididunt          │
│1     │1     │consequat           │
│1     │1     │voluptate           │
│1     │1     │excepteur           │
│1     │1     │cupidatat           │
│1     │1     │pariatur            │
└──────┴──────┴────────────────────┘
┌──────┬────────────────────┐
│Knyga1│Žodis               │
├──────┼────────────────────┤
│1     │duisk1              │
└──────┴────────────────────┘
\end{verbatim}

\begin{changemargin}{0cm}{0cm}
    \textbf{ManoKnyga.txt:}
\end{changemargin}

\inputminted{text}{Assets/L4/t2-mk.txt}

\subsection{Dėstytojo pastabos}
\newpage

%-----------------------------------------------------------------------------------------------------------------
%L5
%-----------------------------------------------------------------------------------------------------------------

\section{Paveldėjimas}
\subsection{Darbo užduotis}
\subsection{Programos tekstas}
\subsection{Pradiniai duomenys ir rezultatai}
\subsection{Dėstytojo pastabos}
\newpage

\end{document}