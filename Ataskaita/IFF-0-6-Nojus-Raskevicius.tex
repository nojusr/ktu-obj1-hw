\documentclass{article}

\usepackage{amsmath}
\usepackage[margin=2cm,a4paper]{geometry} 
\usepackage[T1]{fontenc}
\usepackage[english,lithuanian]{babel}
\usepackage[unicode,hidelinks]{hyperref}
\usepackage{graphicx}
\usepackage{enumitem}
\usepackage[final]{pdfpages}
\usepackage{titlesec}
\usepackage{textcomp}
\usepackage{lmodern, newunicodechar}
\usepackage{minted}
\usepackage{fontspec}

% change subsection formatting
\titleformat*{\subsection}{\large\bfseries}

% small function to set margins for specific elements
\def\changemargin#1#2{\list{}{\rightmargin#2\leftmargin#1}\item[]}
\let\endchangemargin=\endlist 

% set the default fonts
\defaultfontfeatures{Scale = MatchLowercase}
\setmainfont{Liberation Serif}[Scale = 1.0]
\setmonofont{Courier New}

\begin{document}

% redefine matrix command for easily changeable matrix vspacing
\makeatletter
\renewcommand*\env@matrix[1][\arraystretch]{%
  \edef\arraystretch{#1}%
  \hskip -\arraycolsep
  \let\@ifnextchar\new@ifnextchar
  \array{*\c@MaxMatrixCols c}}
\makeatother

% Redefinition of ToC command to get centered heading
\makeatletter
\renewcommand\tableofcontents{%
  \null\hfill\textbf{\Large\contentsname}\hfill\null\par
  \@mkboth{\MakeUppercase\contentsname}{\MakeUppercase\contentsname}%
  \@starttoc{toc}%
}
\makeatother

% minted settings
\setminted{
    breaklines=true,
    fontsize=\small
}




%-------------------------------------
%TITLE PAGE
%-------------------------------------

% ktu logo
\begin{figure}[!ht]
    \centering
    \includegraphics[width=2.5cm]{Assets/ktu-ikona.png}
\end{figure}

\vspace{-0.8cm}

\begin{center}
    \Large\textbf{Kauno technologijos universitetas}
\end{center}

\vspace{-0.6cm}

\begin{center}
    \large{Informatikos fakultetas}
\end{center}

\vspace{5cm}

\begin{center}
    \LARGE\textbf{Objektinis programavimas I (P175B118)}
\end{center}

\vspace{-0.5cm}

\begin{center}
    \Large{Laboratorinių darbų ataskaita}
\end{center}


\vspace{2.5cm}


\begin{changemargin}{7.5cm}{0cm}

    \begin{itemize}
        \item[]\large\textbf{Nojus Raškevičius, IFF-0/6}
        \vspace{0.2cm} 
        \item[]\large{Studentas}
    \end{itemize}

    \vspace{1cm}

    \begin{itemize}
        \item[]\large\textbf{Prof. Vacius Jusas}
        \vspace{0.2cm} 
        \item[]\large{Dėstytojas}
    \end{itemize}

\end{changemargin}

\vspace*{\fill}

\begin{center}
    \large\textbf{Kaunas 2020}
\end{center}

\pagebreak
\newpage

%-----------------------------------------------------------------------------------------------------------------
%CONTENTS
%-----------------------------------------------------------------------------------------------------------------

\tableofcontents

\newpage

%-----------------------------------------------------------------------------------------------------------------
%L1
%-----------------------------------------------------------------------------------------------------------------

\section{Duomenų klasė}
\subsection{Darbo užduotis}
\textbf{Kompiuterinis žaidimas.} Kuriate „fantasy“ tipo kompiuterinį žaidimą. Duomenų faile turite informacija apie žaidimo herojus: vardas, rasė, klasė, gyvybės taškai, mana, žalos taškai, gynybos taškai, jėga, vikrumas, intelektas, ypatinga galia.
\begin{itemize}
    \item[•]Raskite daugiausiai gyvybės taškų turintį herojų, ekrane atspausdinkite jo vardą, rasę, klasę ir gyvybės taškų kiekį. Jei yra keli, spausdinkite visus.
    \item[•]Raskite žaidėją, kurio gynybos ir žalos taškų skirtumas yra mažiausias. Atspausdinkite informaciją apie žaidėją į ekraną. Jei yra keli, spausdinkite visus. 
    \item[•]Sudarykite visų herojų klasių sąrašą, klasių pavadinimus įrašykite į failą „Klasės.csv“ .
\end{itemize}


\subsection{Programos tekstas}

\inputminted{csharp}{Assets/L1/L1-rawtext.txt}

\subsection{Pradiniai duomenys ir rezultatai}

\textbf{Pradiniai duomenys:}

\inputminted{csharp}{Assets/L1/herojai.csv}

\normalsize
\begin{changemargin}{0cm}{0cm}
    \textbf{Rezultatai:}
\end{changemargin}

\footnotesize
\begin{verbatim}
Visi herojai:
┌──────────┬──────────────┬───────────┬────┬────┬────┬─────┬──┬──┬──┬────────────────┐
│Vardas    │Rasė          │Klasė      │G.t.│M.t.│Ž.t.│Gy.t.│J.│V.│I.│Ypat. galia     │
├──────────┼──────────────┼───────────┼────┼────┼────┼─────┼──┼──┼──┼────────────────┤
│Aloyzas   │Lokys         │Kunigas    │97  │72  │38  │35   │7 │7 │1 │Gerai gamina... │
├──────────┼──────────────┼───────────┼────┼────┼────┼─────┼──┼──┼──┼────────────────┤
│Aloyzas   │Žmogus        │Dainininkas│42  │69  │82  │73   │4 │5 │8 │Labai laimin... │
├──────────┼──────────────┼───────────┼────┼────┼────┼─────┼──┼──┼──┼────────────────┤
│Antanas   │Varliažmogis  │Kunigas    │18  │27  │25  │51   │1 │9 │9 │Labai gražio... │
├──────────┼──────────────┼───────────┼────┼────┼────┼─────┼──┼──┼──┼────────────────┤
│Antanas   │Varliažmogis  │Kunigas    │66  │87  │99  │25   │4 │2 │4 │Ugnies valdy... │
├──────────┼──────────────┼───────────┼────┼────┼────┼─────┼──┼──┼──┼────────────────┤
│Petras    │Driežažmogis  │Burtininkas│21  │55  │20  │17   │5 │1 │1 │Labai laimin... │
├──────────┼──────────────┼───────────┼────┼────┼────┼─────┼──┼──┼──┼────────────────┤
│Vardėnis  │Driežažmogis  │Magas      │59  │33  │40  │55   │0 │0 │8 │Labai gražio... │
├──────────┼──────────────┼───────────┼────┼────┼────┼─────┼──┼──┼──┼────────────────┤
│Motiejus  │Elfas         │Lankininkas│35  │44  │45  │40   │3 │7 │5 │Labai laimin... │
├──────────┼──────────────┼───────────┼────┼────┼────┼─────┼──┼──┼──┼────────────────┤
│Motiejus  │Tamsusis elfas│Kunigas    │86  │70  │26  │69   │8 │9 │7 │Labai gražio... │
├──────────┼──────────────┼───────────┼────┼────┼────┼─────┼──┼──┼──┼────────────────┤
│Motiejus  │Elfas         │Karys      │57  │71  │95  │51   │4 │7 │2 │Labai laimin... │
├──────────┼──────────────┼───────────┼────┼────┼────┼─────┼──┼──┼──┼────────────────┤
│Antanas   │Driežažmogis  │Riteris    │33  │6   │91  │80   │3 │4 │7 │Ugnies valdy... │
├──────────┼──────────────┼───────────┼────┼────┼────┼─────┼──┼──┼──┼────────────────┤
│Vardėnis  │Lokys         │Gydytojas  │18  │34  │43  │12   │1 │3 │5 │Ugnies valdy... │
├──────────┼──────────────┼───────────┼────┼────┼────┼─────┼──┼──┼──┼────────────────┤
│Motiejus  │Tamsusis elfas│Lankininkas│75  │63  │18  │22   │1 │2 │7 │Labai laimin... │
├──────────┼──────────────┼───────────┼────┼────┼────┼─────┼──┼──┼──┼────────────────┤
│Juozas    │Šuo           │Karys      │12  │56  │86  │38   │6 │7 │4 │Gerai gamina... │
├──────────┼──────────────┼───────────┼────┼────┼────┼─────┼──┼──┼──┼────────────────┤
│Vardėnis  │Varliažmogis  │Burtininkas│37  │47  │14  │75   │1 │6 │6 │Gerai gamina... │
├──────────┼──────────────┼───────────┼────┼────┼────┼─────┼──┼──┼──┼────────────────┤
│Motiejus  │Žmogus        │Riteris    │28  │23  │61  │81   │3 │1 │9 │Labai gražio... │
├──────────┼──────────────┼───────────┼────┼────┼────┼─────┼──┼──┼──┼────────────────┤
│Antanas   │Elfas         │Riteris    │56  │4   │16  │91   │9 │7 │5 │Labai laimin... │
├──────────┼──────────────┼───────────┼────┼────┼────┼─────┼──┼──┼──┼────────────────┤
│Juozas    │Žmogus        │Karys      │97  │17  │74  │69   │5 │7 │5 │Ugnies valdy... │
├──────────┼──────────────┼───────────┼────┼────┼────┼─────┼──┼──┼──┼────────────────┤
│Motiejus  │Varliažmogis  │Magas      │39  │74  │21  │31   │7 │5 │0 │Labai gražio... │
├──────────┼──────────────┼───────────┼────┼────┼────┼─────┼──┼──┼──┼────────────────┤
│Petras    │Žmogus        │Riteris    │46  │64  │92  │83   │8 │6 │9 │Labai laimin... │
├──────────┼──────────────┼───────────┼────┼────┼────┼─────┼──┼──┼──┼────────────────┤
│Vardėnis  │Elfas         │Karys      │61  │77  │81  │26   │1 │1 │9 │Gerai gamina... │
├──────────┼──────────────┼───────────┼────┼────┼────┼─────┼──┼──┼──┼────────────────┤
│Petras    │Driežažmogis  │Gydytojas  │92  │35  │99  │37   │3 │4 │0 │Labai gražio... │
├──────────┼──────────────┼───────────┼────┼────┼────┼─────┼──┼──┼──┼────────────────┤
│Juozas    │Driežažmogis  │Riteris    │32  │60  │48  │83   │7 │6 │6 │Labai laimin... │
├──────────┼──────────────┼───────────┼────┼────┼────┼─────┼──┼──┼──┼────────────────┤
│Aloyzas   │Driežažmogis  │Lankininkas│65  │21  │98  │68   │6 │6 │9 │Labai gražio... │
├──────────┼──────────────┼───────────┼────┼────┼────┼─────┼──┼──┼──┼────────────────┤
│Juozas    │Driežažmogis  │Gydytojas  │91  │51  │63  │23   │4 │5 │4 │Ugnies valdy... │
├──────────┼──────────────┼───────────┼────┼────┼────┼─────┼──┼──┼──┼────────────────┤
│Juozas    │Zombis        │Gydytojas  │93  │95  │82  │23   │9 │9 │5 │Labai gražio... │
└──────────┴──────────────┴───────────┴────┴────┴────┴─────┴──┴──┴──┴────────────────┘
Herojai su didžiausiu kiekiu gyvybės taškų:
┌──────────────────┬──────────────────┬──────────────────┬──────────────────┐
│ Vardas           │ Rasė             │ Klasė            │ Gyvybės t.       │
├──────────────────┼──────────────────┼──────────────────┼──────────────────┤
│ Aloyzas          │ Lokys            │ Kunigas          │ 97               │
├──────────────────┼──────────────────┼──────────────────┼──────────────────┤
│ Juozas           │ Žmogus           │ Karys            │ 97               │
└──────────────────┴──────────────────┴──────────────────┴──────────────────┘
Herojai su mažiausiu skirtumu tarp žalos ir gynybos taškų:
┌──────────┬──────────────┬───────────┬────┬────┬────┬─────┬──┬──┬──┬────────────────┐
│Vardas    │Rasė          │Klasė      │G.t.│M.t.│Ž.t.│Gy.t.│J.│V.│I.│Ypat. galia     │
├──────────┼──────────────┼───────────┼────┼────┼────┼─────┼──┼──┼──┼────────────────┤
│Aloyzas   │Lokys         │Kunigas    │97  │72  │38  │35   │7 │7 │1 │Gerai gamina... │
├──────────┼──────────────┼───────────┼────┼────┼────┼─────┼──┼──┼──┼────────────────┤
│Petras    │Driežažmogis  │Burtininkas│21  │55  │20  │17   │5 │1 │1 │Labai laimin... │
└──────────┴──────────────┴───────────┴────┴────┴────┴─────┴──┴──┴──┴────────────────┘
\end{verbatim}

\normalsize
\begin{changemargin}{0cm}{0cm}
    \textbf{Klasės.csv:}
\end{changemargin}
\inputminted{csharp}{Assets/L1/klases.csv}

\subsection{Dėstytojo pastabos}
\begin{itemize}
    \item[•]Ataskaitos pavadinimas buvo ne pagal taisykles
    \item[•]Dėstytotojo pareigos buvo ne tos
    \item[•]Metodai nebuvo tinkamai bei visiškai aprašyti
    \item[•]Nėra pateikti komentarai apie pradinių duomenų prasmę
\end{itemize}

\textbf{Galutinis pažymys: 8}

\newpage

%-----------------------------------------------------------------------------------------------------------------
%L2
%-----------------------------------------------------------------------------------------------------------------


\section{Skaičiavimų klasė}

\subsection{Darbo užduotis}
\textbf{Krepšinio rinktinė.} urite  ne  tik  šių,  bet  ir  vienų  ankstesniųjų  metų  į  stovyklas  pakviestų krepšininkų sąrašus. Keičiasi duomenų failų formatas. Pirmoje eilutėje metai, antroje –stovyklos pradžios data, trečioje –stovyklos pabaigos data. Toliau informacija apie krepšininkus pateikta tokiu pačiu formatu kaip L1 užduotyje.
\begin{itemize}
    \item[•]Sudarykite visų puolėjų, dalyvavusių rinktinės stovyklose, sąrašą ir ekrane atspausdinkite jų vardus, pavardes bei ūgį.
    \item[•]Raskite aukščiausiąkrepšininką, ir ekrane atspausdinkite jo vardą, pavardę bei amžių. Jei yra keli, spausdinkite visus.
    \item[•]Sudarykite sąrašą klubų, kuriuose žaidė kandidatai į rinktinę, ir įrašykite į failą „Klubai.csv“.
\end{itemize}

\subsection{Programos tekstas}
\inputminted{csharp}{Assets/L2/L2-rawtext.txt}
\newpage

\subsection{Pradiniai duomenys ir rezultatai}

\subsubsection{Pirmas tikrinimas}
\begin{changemargin}{0cm}{0cm}
    \textbf{krepsininkai-2020.csv:}
\end{changemargin}

\inputminted{csharp}{Assets/L2/test1-input1.txt}

\begin{changemargin}{0cm}{0cm}
    \textbf{krepsininkai-2019.csv:}
\end{changemargin}

\inputminted{csharp}{Assets/L2/test1-input2.txt}


\begin{changemargin}{0cm}{0cm}
    Šitie duomenys yra skirti bendram tikrinimui atlikti. Išvestyje turėtų būti tik vienas aukščiausias žaidėjas,
    (Juozas Pavardėnis), turėtų būti 7 puolėjai, turėtų būti 4 unikalūs klubai (Šaulys, L. Rytas, Žalgiris, Kruojos).
\end{changemargin}

\begin{changemargin}{0cm}{0cm}
    \textbf{Programos išvestis:}
\end{changemargin}


\inputminted[fontsize=\footnotesize]{csharp}{Assets/L2/test1-output.txt}

\footnotesize
\begin{verbatim}
Visi puolėjai:
┌──────────┬──────────────┬───┐
│Vardas    │Pavardė       │Au.│
├──────────┼──────────────┼───┤
│Aloyzas   │Valančiūnas   │182│
├──────────┼──────────────┼───┤
│Petras    │Pavardėnis    │197│
├──────────┼──────────────┼───┤
│Antanas   │Žukauskas     │198│
├──────────┼──────────────┼───┤
│Antanas   │Jasikevičius  │195│
├──────────┼──────────────┼───┤
│Aloyzas   │Valančiūnas   │195│
├──────────┼──────────────┼───┤
│Antanas   │Valančiūnas   │187│
├──────────┼──────────────┼───┤
│Juozas    │Pavardėnis    │207│
└──────────┴──────────────┴───┘
Aukščiausi žaidėjai:
┌──────────┬──────────────┬───┐
│Vardas    │Pavardė       │Au.│
├──────────┼──────────────┼───┤
│Juozas    │Pavardėnis    │207│
└──────────┴──────────────┴───┘
\end{verbatim}
\normalsize

\begin{changemargin}{0cm}{0cm}
    \textbf{Klubai.csv:}
\end{changemargin}

\footnotesize
\begin{verbatim}
L. Rytas
Šaulys
Žalgiris
Kruojos
\end{verbatim}
\normalsize

\subsubsection{Antras tikrinimas}
\begin{changemargin}{0cm}{0cm}
    \textbf{krepsininkai-2020.csv:}
\end{changemargin}

\inputminted{csharp}{Assets/L2/test2-input1.txt}

\begin{changemargin}{0cm}{0cm}
    \textbf{krepsininkai-2019.csv:}
\end{changemargin}

\inputminted{csharp}{Assets/L2/test2-input2.txt}


\begin{changemargin}{0cm}{0cm}
    Šie įvesties duomenys yra skirti tikrinti „Klubai.csv“ išvestį.
    Yra įvesti tokie patys klubai kaip ir pirmajame tikrinime, tačiau jie yra pakeisti taip,
    kad žaidėjai yra priimami tik iš „Žalgirio“ ir „Šaulio“ klubų.
    Kitaip tariant, Į „Klubai.csv“ turėtų buti išvesti tik du klubai: „Šaulys“ ir „Žalgiris“.
\end{changemargin}

\begin{changemargin}{0cm}{0cm}
    \textbf{Programos išvestis:}
\end{changemargin}


\inputminted[fontsize=\footnotesize]{csharp}{Assets/L2/test2-output.txt}

\footnotesize
\begin{verbatim}
Visi puolėjai:
┌──────────┬──────────────┬───┐
│Vardas    │Pavardė       │Au.│
├──────────┼──────────────┼───┤
│Vardėnis  │Sabonis       │193│
├──────────┼──────────────┼───┤
│Antanas   │Valančiūnas   │191│
├──────────┼──────────────┼───┤
│Juozas    │Pavardėnis    │194│
├──────────┼──────────────┼───┤
│Petras    │Sabonis       │191│
└──────────┴──────────────┴───┘
Aukščiausi žaidėjai:
┌──────────┬──────────────┬───┐
│Vardas    │Pavardė       │Au.│
├──────────┼──────────────┼───┤
│Petras    │Pavardėnis    │209│
├──────────┼──────────────┼───┤
│Antanas   │Sabonis       │209│
└──────────┴──────────────┴───┘
\end{verbatim}
\normalsize

\begin{changemargin}{0cm}{0cm}
    \textbf{Klubai.csv:}
\end{changemargin}

\footnotesize
\begin{verbatim}
Šaulys
Žalgiris
\end{verbatim}
\normalsize

\subsection{Dėstytojo pastabos}
\newpage

%-----------------------------------------------------------------------------------------------------------------
%L3
%-----------------------------------------------------------------------------------------------------------------

\section{Konteineris}
\subsection{Darbo užduotis}
\subsection{Programos tekstas}
\subsection{Pradiniai duomenys ir rezultatai}
\subsection{Dėstytojo pastabos}
\newpage

%-----------------------------------------------------------------------------------------------------------------
%L4
%-----------------------------------------------------------------------------------------------------------------

\section{Teksto analizė ir redagavimas}
\subsection{Darbo užduotis}
\subsection{Programos tekstas}
\subsection{Pradiniai duomenys ir rezultatai}
\subsection{Dėstytojo pastabos}
\newpage

%-----------------------------------------------------------------------------------------------------------------
%L5
%-----------------------------------------------------------------------------------------------------------------

\section{Paveldėjimas}
\subsection{Darbo užduotis}
\subsection{Programos tekstas}
\subsection{Pradiniai duomenys ir rezultatai}
\subsection{Dėstytojo pastabos}
\newpage

\end{document}